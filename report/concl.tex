\addcontentsline{toc}{section}{Заключение}
\section*{Заключение}
По результатам проделанной работы можно утверждать, что ФРП обладает рядом важных достоинств.
Программы, получаемые при разработке с применением функционального реактивного программирования,
просты для понимания, поддержки и усовершенствования, поскольку основную часть кода программы составляет
\emph{декларативное} описание производимых вычислений. Исходный код достаточно точно, лаконично и
наглядно описывает модель, на которой основывается логика приложения, что заметно упрощает
понимание принципов его работы, анализ, сопровождение и, при необходимости, модификацию.

Кроме того, использование строго типизированного чисто функционального языка программирования дает
дополнительные гарантии работоспособности и надежности программы, а также значительно упрощает
отладку, поскольку многие ошибки могут быть обнаружены до запуска программы
во время проверки типов.

\addcontentsline{toc}{subsection}{Дальнейшее развитие проекта}
\subsection*{Дальнейшее развитие проекта}
Ввиду отсутствия библиотек на языке программирования \emph{Haskell}, реализующих в достаточной степени протокол XMPP,
имеет смысл расширить текущую реализацию, добавив статусы, а также видео-связь.
Полученная библиотека может распространяться отдельно от приложения.
Само приложение может расширяться по мере развития библиотеки.

Исходный код приложения доступен по адресу \url{https://github.com/ademinn/reactive-jabber}.
Данное программное обеспечение является свободным и распространяется на условиях лицензии BSD.
