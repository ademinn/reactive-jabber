\section{Реализация}
\subsection{Инструменты}
Для реализации был выбран язык программирования \emph{Haskell} и библиотека реактивного программирования \emph{reactive-banana}.
Для создания графического интерфейса приложения использовалась библиотека \emph{gtk2hs} (основанная на \emph{GTK+}).

\subsection{Модель}
При разработке с применением ФРП в первую очередь необходимо было построить модель, описывающую
игру, выделить требуемые события и сигналы, определить зависимости между ними.

Внутренним состоянием интерфейса является информация об активных чатах, и история переписки в них.
Такую информацию удобно представить в виде отображения из JID собеседников в структуру, хранящую необходимую информацию о чате.
Данное отображение представляется в виде кусочно-постоянного сигнала, изначально равного пустому отображению.
Добавление нового собеседника происходит в двух случаях:
\begin{itemize}
    \item от собеседника приходит сообщение;
    \item пользователь двойным кликом выбирает собеседника из своего списка контактов.
\end{itemize}
Для этих двух случаев созданы отдельные события,
первое из которых наступает при поступлении соответствующей информации от библиотеки для работы с XMPP,
а наступление второго управляется при помощи gtk2hs.
На основе этих двух событий строится итоговое событие, наступление которого обозначает, что необходимо добавить нового собеседника.
Добавленные собеседники не удаляются из отображения до конца работы программы.

Полный список событий и сигналов привден в подразделе~\ref{events_and_signals}

\subsection{События и сигналы}\label{events_and_signals}
Интерфейс приложения управляется сетью событий.
К ее входам подключены следующие события:
\begin{itemize}
    \item eInMsg --- новое сообщение от сервера:
    \begin{itemize}
        \item входящее сообщение;
        \item запрос на авторизацию;
        \item подтверждение авторизации;
        \item отказ авторизации;
    \end{itemize}
    \item eOutMsg --- новое исходящее сообщение;
    \item eShowChat --- двойной клик пользователя на одном из элементов списка контактов.
\end{itemize}
Выводами сети логики являются следующие события:
\begin{itemize}
    \item добавить новый чат;
    \item открыть вкладку с уже существующим чатом;
    \item отобразить новое сообщение в существующем чате;
    \item отправить сообщение определенному собеседнику;
    \item отправить ответ пользователя на запрос авторизации;
    \item добавить новый элемент в список контактов;
    \item удалить элемент из списка контактов;
\end{itemize}


\subsection{Исходный код}
\input{reactive-jabber.tex}
\input{Parser.tex}
\input{XMPPTypes.tex}
\input{XMPPMapping.tex}
\input{XMPP.tex}