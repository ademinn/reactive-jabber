\section{Реализация}
\subsection{Инструменты}
Для реализации был выбран язык программирования \emph{Haskell} и библиотека реактивного программирования \emph{reactive-banana}.
Для создания графического интерфейса приложения использовалась библиотека \emph{gtk2hs} (основанная на \emph{GTK+}).

\subsection{Модель}
При разработке с применением ФРП в первую очередь необходимо было построить модель, описывающую
игру, выделить требуемые события и сигналы, определить зависимости между ними.

Внутренним состоянием интерфейса является информация об активных чатах, и история переписки в них.
Такую информацию удобно представить в виде отображения из JID собеседников в структуру, хранящую необходимую информацию о чате.
Данное отображение удобно представить в виде кусочно-постоянного сигнала, изначально равного пустому отображению.
Добавление нового элемента в отображение происходит, когда либо приходит сообщение, либо пользователь кликает на

\subsection{Исходный код}
\input{reactive-jabber.tex}
\input{Parser.tex}
\input{XMPPTypes.tex}
\input{XMPPMapping.tex}
\input{XMPP.tex}